
%   \btypeout{Abstract Page}
\thispagestyle{empty}
\null\vfil
\begin{center}
{\huge{\textit{Resumo}} \par}
\bigskip
% {\normalsize\bf \@title \par}
\medskip
\bigskip
\end{center}

Representações tradicionais de mídia digital baseiam-se em amostragem discreta, o que
requer métodos custosos de interpolação para calcular resoluções variáveis. As redes neurais,
por outro lado, oferecem uma representação contínua que aproxima modelos matemáticos de
sinais à implementação computacional. Avanços recentes em aprendizado profundo
introduziram redes neurais baseadas em coordenadas para aplicações como campos neurais
ou representações neurais implícitas, que aproveitam essa perspectiva. Com base nesses
avanços, investigamos a aplicação da teoria de multirresolução a redes neurais baseadas em
coordenadas para abordar desafios na representação de mídia neural, incluindo zoom, anti-
aliasing e transmissão eficiente de dados.

Através de um estudo sistemático e fundamentado do aprendizado de frequência em redes
com funções de ativação sinusoidal, desenvolvemos a MR-Net, uma família de redes neurais
para codificar objetos de mídia em múltiplas escalas. Além disso, propomos uma inicialização
inspirada na Série de Fourier, provando que ela restringe as redes neurais sinusoidais a
espaços de funções periódicas, e mostramos como isso possibilita novas operações, como a
otimização de texturas de materiais sem costuras visíveis, através de um termo de
regularização baseado na equação de Poisson. Demonstramos o potencial da MR-Net na
codificação de imagens, anti-aliasing, mapeamento de texturas e síntese de materiais sem
costuras. Além disso, em tarefas de reconstrução de imagens, a MR-Net alcança valores de
Relação Sinal-Ruído de Pico (PSNR) comparáveis ou superiores a modelos anteriores de
última geração, usando menos parâmetros. Esta tese contribui, portanto, para a conexão entre
a teoria de multirresolução e as redes neurais, sugerindo direções práticas e escaláveis para
aplicações contemporâneas de mídia, inclusive com objetos mais complexos.