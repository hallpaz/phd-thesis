\chapter{Interesting Directions}
\label{chap:future}


% % \subsubsection{Sampling}
% % The input for the neural network is a discrete representation. In this sense,} sampling is a way to create a representation of a function based on its values at certain points of the domain. From the point of view of Representation Theory, we can interpret this as a projection of the function onto the primal Shannon basis (i.e., Dirac delta distribution). In the context of signal processing, this basis is a sampling grid of impulses and the representation consists of the sequence of signal values at the grid locations.

% % One important aspect of sampling for learning signals with coordinate-based neural network is the structure of the sampling grid. In that respect, it is instrumental to consider two types of grid structures: Regular and Irregular. See Fig~\ref{f:sampling}. {\color{red}The sampling mode is stored in the MR-Structure, so it can be taken into account when doing operations that modify the sampling grid.}

% % \begin{figure}[!h]
% % \centering
% % \includegraphics[width=0.42\linewidth]{img/ch4/regular.png}
% % \hfil
% % \includegraphics[width=0.42\linewidth]{img/ch4/poisson.png}
% % \\
% % {\hfil (regular grid) \hfil\hfil (irregular grid) \hfil}
% % \caption{Sampling Modes}
% % \label{f:sampling}
% % \end{figure}


% % \subsubsection{Filtering}

% % The input of signal values to the network can be filtered to separate it into different frequency bands. In this sense, we can use the unfiltered signal, a low-pass version of the signal or a band-pass version of the signal. See Fig~\ref{f:filter}.

% % It is common to employ a Gaussian Kernel as the low-pass Kernel and a Difference of Gaussians as the band-pass kernel.

% % \begin{figure}[!h]
% % \centering
% % \includegraphics[width=0.30\linewidth]{img/ch4/signal.png}
% % \includegraphics[width=0.30\linewidth]{img/ch4/gaussian.png}
% % \includegraphics[width=0.30\linewidth]{img/ch4/laplacian.png}\\
% % {\hfil \hfil signal \hfil \hfil \hfil low-pass \hfil \hfil \hfil band-pass \hfil}
% % \caption{Filter Types}
% % \label{f:filter}
% % \end{figure}

% % \subsubsection{multi-stage stack}

% % A Multiresolution Stack consists of a hierarchy of sampling grids with different resolutions. The standard grid structure form a dyadic lattice of regular grids obeying the $2^j$ rule, i.e., each level of the grid has twice the size of the previous one. 
% % Nonetheless, it is also possible to define a irregular multiresolution grid structure. In this case, each resolution level has approximately twice the number of random sample points of the previous level (see Fig~\ref{f:multi}(b)).
% % On the other hand, it is possible to have a stack of sampling grid with the same resolution, in which the signal is the same at each level or is filtered (see Fig~\ref{f:multi}(a)). Section~\ref{s:lod} gives more details on this option.
% % \begin{figure}[!h]
% % \centering
% % \includegraphics[width=0.45\linewidth]{img/ch4/tower.png}
% % \includegraphics[width=0.45\linewidth]{img/ch4/pyramid.png}\\
% % {\hfil tower \hfil \hfil \hfil pyramid \hfil}
% % \vspace{-0.2cm}
% % \caption{\hl{Multiresolution Hierarchies}}
% % \label{f:multi}
% % \end{figure}