\textit{These are the only words in my mother tongue registered in this media after much blood, sweat and tears.}

Minha vida nunca foi tranquila ou monótona. Com experiências que vão desde rotinas extenuantes de estudo, treinamento militar e, até mesmo, participação ativa em política partidária, nunca faltaram fortes emoções. Mesmo assim, a jornada do doutorado se sagrou o período mais intenso da minha vida (até agora – Homer Simpson). Por mais que tentemos separar a vida pessoal da profissional, somos seres humanos inteiros o tempo todo. Neste momento em que visualizo os altos e baixos em perspectiva - ironicamente, em uma tese sobre funções sinusoidais - avalio que o saldo foi positivo, graças ao apoio essencial de muitas pessoas. Abaixo, um singelo agradecimento àqueles que contribuíram para que eu superasse as dificuldades e alcançasse os resultados de que tanto me orgulho.

Gostaria de começar agradecendo ao meu orientador, professor Luiz Velho, pela parceria de longa data, pela construção de um ambiente de colaboração saudável e produtivo para que este trabalho fosse desenvolvido da melhor maneira possível e pelo empenho em me ajudar em diversos momentos, indo muito além do que o senso comum esperaria de um orientador. Luiz é um exemplo de líder que enxerga a vida em suas complexidades e os profissionais como seres humanos. Agradeço também aos colegas do laboratório Visgraf e membros ou ex-membros do grupo Neural Media, pelas discussões profícuas e estimulantes para esta e outras pesquisas, tornando a jornada de doutorado menos solitária. Nominalmente: Alberto Kopiler, Daniel Perazzo, Diana Aldana, Fábio Suim, Guilherme Schardong, Hélio Lopes, Luiz Schirmer, Tiago Novello e Vinícius Silva.

Agradeço à minha família pelo apoio e pela torcida constante, mesmo que muitas vezes não compreendessem exatamente o que eu estava fazendo. Agradeço especialmente à minha mãe e ao meu pai, que realmente foram meu porto seguro nos vales desta caminhada. Agradeço à minha irmã e ao meu irmão, que, além de compartilharem momentos de descontração em família, também deram alguns passos na academia, fortalecendo ainda mais nosso vínculo ao compartilharmos desafios e conquistas.

Agradeço à Kizzy, companheira que esteve ao meu lado em marcos importantíssimos deste processo, com acolhimento e sabedoria. Seu apoio no período do exame de qualificação, na busca de um estágio internacional, na escrita do primeiro artigo, dentre tantos outros momentos de sobrecarga e dúvida, foi imprescindível. Todas as emoções que vivenciamos juntos desde a candidatura ao doutorado até a defesa da tese, são um marco por si só neste processo intenso.

Agradeço à Parris pelo carinho e pelos momentos de leveza e alegria em meio aos desafios do doutorado. Seu apoio antes do CVPR e do SIGGRAPH tornou minha participação nessas conferências algo ainda mais incrível. Sua presença em minha vida impulsionou a minha jornada de autoconhecimento e descobertas. Sou grato por tudo o que compartilhamos.

Esta tese realmente tem um pedaço de muitas pessoas e de muitos lugares. Foi escrita no Rio de Janeiro, em Pittsburgh, em São Paulo, em Salvador e em Seattle; em apartamentos, casas, cafés, bibliotecas, hotéis, aeroportos e aviões. Até mesmo em uma cama de hospital. Cada um desses espaços carrega lembranças de desafios, conquistas e o apoio de pessoas sem as quais este trabalho não seria possível.